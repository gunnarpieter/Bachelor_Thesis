\chapter{Introduction}\label{introduction}
\begin{comment}

The introduction of your bachelor thesis introduces the research area, the
research hypothesis, and the scientific contributions of your work.
A good narrative structure is the one suggested by Simon Peyton Jones
\cite{peys04:HowToWriteAGoodResearchPaper}:
%

\begin{itemize}
\item describe the problem / research question
\item motivate why this problem must be solved
\item demonstrate that a (new) solution is needed
\item explain the intuition behind your solution
\item motivate why / how your solution solves the problem (this is technical)
\item explain how it compares with related work
\end{itemize}
%
Close the introduction with a paragraph in which the content of the next chapters
is briefly mentioned (one sentence per chapter).
\end{comment}

% An increase in the surface temperature of our planet has been observed. The Intergovernmental Panel on Climate Change (IPCC) concludes that "it is extremely likely that human influence has been the dominant cause of the observed warming since the mid-20th century"\footnote{\url{https://archive.ipcc.ch/pdf/assessment-report/ar5/wg1/WG1AR5_SPM_FINAL.pdf}}. 
% Multiple countries have tried to make agreements regarding the emission of the greenhouse gasses that cause this increase. An example of this is the Paris Climate Agreement.
% Within the Netherlands the government also made multiple year agreements with the industry to reduce the burden of our industry on the environment. The new goals state that the ICT sector should be 50\% more energy efficient in 2030 compared to 2005 \footnote{\url{https://www.nldigital.nl/news/routekaart-ict-2030-van-start-gegaan/}}.\\
% \\
%Since the creation of the internet, it has seen nonstop expansion with an estimate of 4131$\sim$4536 million users in 2019 \footnote{\url{https://www.statista.com/statistics/273018/number-of-internet-users-worldwide/}} \footnote{\url{https://www.internetworldstats.com/stats.htm}}.
Within the Netherlands alone 98\% of the households have an internet connection. In those households, smartphones and laptops are the most common devices to surf the web with\footnote{\url{https://longreads.cbs.nl/ict-kennis-en-economie-2019/ict-gebruik-van-huishoudens-en-personen/}}. Next to this,  WiFi networks are generally used for longer sessions transferring larger amounts of data than their cellular counter part \cite{5401061}.
WiFi capable devices like smartphones and laptops are also known as stations (STAs) and use WiFi (IEEE 802.11) to connect with an access point (AP). The AP is a gateway that allows the devices to be wireless but still maintain access to the internet, transmitting and receiving the packets to and from stations using a specific version of the standard with a set bandwidth. The bandwidth is a specific range of frequencies that the AP uses to transfer data on and in this case the bandwidth can either be 20, 40 or 80MHz. A larger bandwidth allows for more data to be transferred but also has a higher risk of interference with other WiFi devices. More and more of these networks are being installed. 
%The electricity consumption growth rate of communication networks, personal computers, and data centers has been estimated at 10\%, 5\% and 4\% respectively. This is higher than the estimated overall energy growth rate of 3\% \cite{VANHEDDEGHEM201464}. 
In 2007 communication networks, personal computers, and data centers used 3.9\% of the total energy consumption worldwide, in 2012 this grew to 4.6\% \cite{VANHEDDEGHEM201464}.
From the total amount of electricity consumed by ICT a significant 29\% is consumed solely by networks \cite{ClickClean2016}.

%The Institute of Electrical and Electronics Engineers (IEEE) develops both the WiFi and Ethernet (802.3) standards.\\
%"In an attempt to make the internet more energy efficient IEEE developed enhancements for the ethernet standard, Energy Efficient Ethernet 802.3az\cite{5621967}."\todo{Niet in intro deze zin maar in background wellicht wel?} 
Within 802.11 only stations have the option to enter an energy efficient state. In this state the STA tells the AP that it will go into Power Saving Mode, the AP buffers the packets that arrive for the STA and when the AP broadcasts its beacon. The beacon includes which STAs have buffered packets waiting, such that those STAs can then retrieve their packets.\\
\\
Different studies have been performed to make WiFi networks more energy efficient. Examples of these studies can be found in chapter \ref{relatedwork}. 
%An example of this can be found in \cite{6970705} where an AP's connections are offloaded to another AP when the total load on both APs allows this. 
Next to this, one could argue that ZigBee (802.15.4) or Bluetooth Low Energy (802.15.1) can be used instead because they are more energy efficient compared to WiFi. However, ZigBee has a maximum transfer rate of 250 Kbps\cite{FARAHANI20081} and Bluetooth without using an 802.11 link can only achieve 2Mbps \cite{8419192}. Transferring the average European monthly data consumption of 175.7GB \cite{OpenVaultReport2019} would take ZigBee and Bluetooth, 68 and 8 days respectively.
Only a 4G or higher cellular internet connection can compete with WiFi providing between $100 \sim 1000$Mbps \cite{7724643} versus the 802.11ac standard achieving between 433 Mbps and several Gbps\footnote{https://www.actiontec.com/wifihelp/evolution-wi-fi-standards-look-802-11abgnac/}.
%Where 802.11n has a maximum transfer rate of 300 Mbps which is 1200 times more \footnote{https://www.actiontec.com/wifihelp/evolution-wi-fi-standards-look-802-11abgnac/}. 
WiFi is vastly superior in data transfer rates compared to ZigBee and Bluetooth and similar to cellular, however greenifying WiFi access points is falling behind.
% \subsection{Research question}\todo{Research question}
Access points are not optimized to conserve energy. This leads to the main research questions: 
\begin{quotation}
    \noindent
    \textbf{What are the possibilities of increasing the energy efficiency of an Archer C7 AC1750 access point using the WiFi standards b/a/g/n/ac?}
\end{quotation}
In other words, the main goal is to investigate if the energy consumption of an Archer C7 AC1750 access point can be reduced through the use of the WiFi standards b/a/g/n/ac.\\
The energy consumption of only one AP will be measured using three files with three runs each. This, is done within a constant interferenceless environment. The energy consumption results of the small scale measurements will be extrapolated to calculate the energy consumption over a longer period of a month within the average home. We will not produce our own WiFi standard nor perform monthlong measurements.\\
This thesis contributes, the measurement results, extrapolations of those results to a period of a month, the measurement setup and the code to perform the measurements. Next to this, recommendations are done regarding the most energy efficient WiFi standard, and regarding standard switching compared to a wake/sleep cycle.\\
To achieve the goal the energy consumption of the access point needs to be known. The measured energy consumption for transmitting and the power draw when idling are discussed in chapter \ref{researchExperiment}. Section \ref{MostEnergyEffStandard} provides and analyses the AP's energy consumption of a month, extrapolated from the results of the experiment. Next to this, in section \ref{section:SwitchingDynamically} switching between the standards over time to reduce energy consumption over the period of a month is simulated. Chapter \ref{relatedwork} covers related approaches to the problem at hand. Finally, in chapter \ref{conclusions} conclusions are drawn from the results and future work is discussed.


% \section{Contribution}
%\section{Structure}
