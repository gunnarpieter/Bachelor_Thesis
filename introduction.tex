\section{Introduction}\label{introduction}
\begin{comment}

The introduction of your bachelor thesis introduces the research area, the
research hypothesis, and the scientific contributions of your work.
A good narrative structure is the one suggested by Simon Peyton Jones
\cite{peys04:HowToWriteAGoodResearchPaper}:
%

\begin{itemize}
\item describe the problem / research question
\item motivate why this problem must be solved
\item demonstrate that a (new) solution is needed
\item explain the intuition behind your solution
\item motivate why / how your solution solves the problem (this is technical)
\item explain how it compares with related work
\end{itemize}
%
Close the introduction with a paragraph in which the content of the next chapters
is briefly mentioned (one sentence per chapter).
\end{comment}

An increase in the surface temperature of our planet has been observed. The Intergovernmental Panel on Climate Change (IPCC) concludes that "it is extremely likely that human influence has been the dominant cause of the observed warming since the mid-20th century"\footnote{\url{https://archive.ipcc.ch/pdf/assessment-report/ar5/wg1/WG1AR5_SPM_FINAL.pdf}}. 
Multiple countries have tried to make agreements regarding the emission of the greenhouse gasses that cause this increase. An example of this is the Paris Climate Agreement.
Within the Netherlands the government also made multiple year agreements with the industry to reduce the burden of our industry on the environment. The new goals state that the ICT sector should be 50\% more energy efficient in 2030 compared to 2005 \footnote{\url{https://www.nldigital.nl/news/routekaart-ict-2030-van-start-gegaan/}}.\\
\\
%Since the creation of the internet, it has seen nonstop expansion with an estimate of 4131$\sim$4536 million users in 2019 \footnote{\url{https://www.statista.com/statistics/273018/number-of-internet-users-worldwide/}} \footnote{\url{https://www.internetworldstats.com/stats.htm}}.
Within the Netherlands alone 98\% of the households have an internet connection. In those households, smartphones and laptops are the most common devices to surf the web with \cite{ICTUsageCBS}. Next to this,  WiFi networks are generally used for longer sessions transfering larger amounts of data than their cellular counter part \cite{5401061}. When talking about WiFi smartphones and laptops are also known as stations (STAs) and use WiFi (IEEE 802.11 standard) to connect with an access point (AP). The AP is a gateway that allows the devices to be wireless but still maintain access to the internet, transmitting and receiving the packets to and from the wireless devices. More and more of these networks are being installed. The electricity consumption growth rate of communication networks, personal computers, and data centers has been estimated at 10\%, 5\% and 4\% respectively. This is higher than the estimated overall energy growth rate of 3\% \cite{VANHEDDEGHEM201464}. In 2007 these three used 3.9\% where as in 2012 this was 4.6\%.
Not only is the growth rate of the electricity consumption of networks substantial, but the contribution of networks to the total electricity usage of the ICT sector is also significant with 29\% \cite{ClickClean2016}.

The Institute of Electrical and Electronics Engineers (IEEE) develops both Wi-Fi and Ethernet (802.3) standards.\\
In an attempt to make the internet more energy efficient IEEE developed enhancements for the ethernet standard, namely Energy Efficient Ethernet 802.3az\cite{5621967}. However for the Wi-Fi standard only STAs have the option to enter an energy efficient state. In this state the STA tells the AP that it will go into Power Saving Mode, the AP will buffer the packets that arrive for the STA and when the AP broadcasts its beacon. It will include which STAs have buffered packets. Such that those STAs can then retrieve their packets.\\
\\
Different studies have been done to make Wi-Fi networks more energy efficient. An example of this can be found in \cite{6970705} where an APs connections are offloaded to another AP when the totla load on both APs allow this. Next to this one could argue that ZigBee (802.15-4) can be used instead because it is more energy efficient compared to Wi-Fi however ZigBee has a maximum transfer rate of 250 Kbps\cite{FARAHANI20081}. Where 802.11n has a maximum transfer rate of 300 Mbps which is 1200 times more \footnote{https://www.actiontec.com/wifihelp/evolution-wi-fi-standards-look-802-11abgnac/}. Wi-Fi is thus superior in data transfer rates than ZigBee, however greenifying Wi-Fi access points is falling behind.
Within this paper a comparison of possible options/methods to reduce the energy consumption of APs will be made.


\paragraph{Research question}\todo{Research question}
Kan een dynamisch switchen tussen WiFi standaarden (b/a/g/n/ac) energie besparen?
 What is the power consumption of each  WiFi standards?
 What are 
\subsection{Contribution}
\subsection{Structure}
