\section{Potential energy saving}\label{research2}
This chapter, or series of chapters, delves into all technical details that are
required to \emph{prove} your scientific hypothesis.
It should be sufficiently detailed and precise in order for any fellow computing scientist student to be able to \emph{repeat}
your research and therewith establish the same results / conclusions that you have obtained.
Please note that, in order to improve readability of your thesis, you can put a part of this information also in one or
%more appendices (see Appendix \ref{appendix}).


\subsection{Switching dynamically}\label{section:SwitchingDynamically}
Within this chapter the meaning of dynamically switching will be explained.\newline
\newline
Dynamically switching between the different WiFi standards, is that when the AP is not actively used by the connected stations that the AP would choose a older WiFi standard than the standard the AP is currently using to operate on and switch to that standard. Next to this, as the demand for more and faster data transmission rises that the AP if possible choose a newer WiFi standard. 
The goal of dynamically switching is to decrease the energy consumption by the AP and at the same time maintain usability and user comfort. 
The switch frequency, is not defined. Due to the possibility of a switch endangering either usability of the network or user comfort when the switch is performed when the network is in use.

\subsubsection{Drawbacks switching dynamically}
The technique which we have called dynamically switching or switching dynamically has a few drawbacks compared to operating without a WiFi standard restrictions. 
The most prominent drawbacks are when such a switch occurs the AP can not send or receive data as the AP has to (partially) reboot, the stations will have to go through the steps explained in section \ref{subsection:Network-AUTH-ASS} to re-authenticate and re-associate with the network. Next to this, if the WiFi network is password protected a key exchange must occur. For the experiment done in chapter \ref{researchExperiment} a few dozen of these switches have been performed manually.
These switches took anywhere from 10 to 30 seconds. 

\subsubsection{Verdict}
\begin{enumerate}
    \item explain what switching dynamically is
    \item the difficulties/problem
    \item Conclude that this will not be possible.
\end{enumerate}

\subsection{Efficient WiFi}
In the previous section (\ref{section:SwitchingDynamically}) an option that applies older WiFi standard when possible, has been discussed. 
However, in chapter \ref{researchExperiment} the experiment showed that the newest WiFi standards are most energy efficient. 
This means that another option to allow the AP to operate more energy efficient is to when possible apply the latest WiFi standard when a connection is made or allow only connections to be made with the latest WiFi standard.

\subsubsection{Drawbacks latest WiFi standard}
\begin{enumerate}
    \item Having a preference in the WiFi standard 
        \begin{itemize}
            \item This WiFi network still allows stations to connect with all WiFi standards but prefers the newest standard.
            Stations that are only capable of the oldest b WiFi standard, will force the other stations to claim the medium using a by b stations decodable control frame. 
            After claiming the medium, the other stations can transmit their data using their own WiFi standard.
            When using a hybrid b/g/n WiFi network, this "may cause a significant overhead to network performance" \cite{1460529}.
        \end{itemize}
\end{enumerate}

\begin{enumerate}
    \item explain what options exist
    \item the difficulties/problems
    \item Conclude ?
\end{enumerate}