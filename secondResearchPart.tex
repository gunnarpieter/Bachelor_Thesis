\section{Potential energy saving}\label{research2}
% This chapter, or series of chapters, delves into all technical details that are
% required to \emph{prove} your scientific hypothesis.
% It should be sufficiently detailed and precise in order for any fellow computing scientist student to be able to \emph{repeat}
% your research and therewith establish the same results / conclusions that you have obtained.
% Please note that, in order to improve readability of your thesis, you can put a part of this information also in one or
%more appendices (see Appendix \ref{appendix}).
\todo{Add text here}

\subsection{Switching dynamically}\label{section:SwitchingDynamically}
Within this chapter the meaning of dynamically switching will be explained.\newline
\newline
Dynamically switching between the different WiFi standards, is that when the AP is not actively used by the connected stations that the AP would choose a older WiFi standard than the standard the AP is currently using to operate on and switch to that standard. Next to this, as the demand for more and faster data transmission rises that the AP if possible choose a newer WiFi standard. 
The goal of dynamically switching is to decrease the energy consumption by the AP and at the same time maintain usability and user comfort. 
The switch frequency, is not defined. Due to the possibility of a switch endangering either usability of the network or user comfort when the switch is performed when the network is in use.

\subsubsection{Drawbacks switching dynamically}
The technique which we have called dynamically switching or switching dynamically has a few drawbacks compared to operating without WiFi standard restrictions. 
The most prominent drawbacks are when such a switch occurs the AP can not send or receive data as the AP has to (partially) reboot, the stations will have to go through the steps explained in section \ref{subsection:Network-AUTH-ASS} to re-authenticate and re-associate with the network. Next to this, if the WiFi network is password protected a key exchange must occur. For the experiment done in chapter \ref{researchExperiment} a few dozen of these switches have been performed manually.
These switches took anywhere from 10 to 30 seconds. 

\subsubsection{Outcome}
The results from the experiment performed in section \ref{researchExperiment} have shown older WiFi standards consume more energy when transmitting but less energy when idling. This means that dynamically switching, can possibly lead to a reduction in the energy consumption of the access point.\newline
This reduction should be found not when transmitting data because then the extra time transmitting will vanquishes the small amount of power draw difference seen in figure \ref{fig:results-100mb-power} but when the WiFi points are idling seen in figure \ref{fig:results-idle-power}. Next to this, it is apparent that the switch should also not be between old and new but 2.4GHz and 5GHz. As both n2-HT20 and n2-HT50 have a much lower energy consumption when transmitting compared to b and g. This would combine both a good transmitting energy consumption and an good idle energy consumption.\newline
\newline
Using the results form the experiment in section \ref{researchExperiment}. The energy consumption of an access point can be calculated using the amount of data that needs to be transferred. The weighted amount of data transferred in Europe is 175.7GB per month. Europe has decreased ethernet usage but seems to have the same growth compared to the U.S.A where the weighted average is 273.5 \cite{OpenVaultReport2019} 
In table \ref{table:EnergyConAP} the energy consumption of a month of an AP has been calculated with the access point idling with WiFi on and idling with WiFi off. Very visible is that the newer standards end up consuming more energy due to a higher idling energy consumption. This difference is eliminated when idling means that the WiFi network is off.

% \begin{enumerate}
%     \item explain what switching dynamically is
%     \item the difficulties/problem
%     \item Conclude that this will not be possible.
% \end{enumerate}

\begin{table}[H]
\label{table:EnergyConAP}
\caption{Total number of Whs consumed by an AP per month using the weighted amount of data consumed in 1Q19, sending 100MB chunks }
\centering
\resizebox{\textwidth}{!}{%
\begin{tabular}{lllllll}
       &Wifi standard & \parbox[t]{2cm}{ Wh Wifi \\ transmitting} & Wh WiFi idle & Wh WiFi off & Wh total idle & Wh total off \\\hline\hline
2.4GHz & B        & 105.89               & 1553.55      & 1510.18     & 1659.44    & 1616.06   \\ \cline{2-7} 
       & G        & 71.38                & 1570.54      & 1532.55     & 1641.93    & 1603.93   \\ \cline{2-7} 
       & N2-HT20  & 38.57                & 1597.97      & 1551.80     & 1636.53    & 1590.36   \\ \cline{2-7} 
       & N2-HT40  & 18.79                & 1635.51      & 1562.35     & 1654.30    & 1581.14   \\
       &          &                      &              &             &            &           \\
5GHz   & A        & 53.59                & 1717.93      & 1538.58     & 1771.52    & 1592.17   \\ \cline{2-7} 
       & N5-HT20  & 28.94                & 1734.95      & 1557.29     & 1763.89    & 1586.23   \\ \cline{2-7} 
       & N5-HT40  & 14.65                & 2122.91      & 1564.96     & 2137.56    & 1579.61   \\ \cline{2-7} 
       & AC-VHT20 & 23.34                & 2062.46      & 1558.46     & 2085.80    & 1581.81   \\ \cline{2-7} 
       & AC-VHT40 & 11.75                & 2121.17      & 1565.51     & 2132.91    & 1577.26   \\ \cline{2-7} 
       & AC-VHT80 & 5.81                 & 2306.96      & 1569.16     & 2312.78    & 1574.97   \\ \cline{2-7} 
\end{tabular}%
}
\end{table}

To provide an close approximation on the energy that can be conserve using this strategy, we attempt to calculate the total energy consumption of the access point. For this calculation several assumptions must be made:
\begin{enumerate}
    \item the faster IEEE 802.11ac with its 80MHz bandwidth (AC-VHT80) is used for 5 hours a day to provide superior speeds than any 2.4GHz standard.
    \item Within those 5 hours 50\% of the daily data is transmit (175.7 a month, $\frac{365 \mbox{ days}}{12 \mbox{ months}} \approx 30.42$ days a month thus $\frac{175.7 GB}{30.42 days} \approx 5.78$ GB per day)
    \item Within the remaining 19 hours the other 50\% is transmit.
    \item during the 5 hour period AC-VHT80 is used.
    \item during the 19 hour period IEEE 802.11n on a 2.4GHz frequency (n2-HT20 and n2-HT40) is used.
\end{enumerate}




\begin{table}[H]
\caption{Energy consumption for \% of data for a specified number of hours.}
\resizebox{\textwidth}{!}{%
\begin{tabular}{llllll}
WiFi standard &
  \% of data &
  hours &
  \begin{tabular}[c]{@{}l@{}}Amount of J\\  to send data\end{tabular} &
  \begin{tabular}[c]{@{}l@{}}Amount of J\\  to idle\end{tabular} &
  \begin{tabular}[c]{@{}l@{}}Total per dag\\ comsumption (J)\end{tabular} \\ \hline \hline
AC-VHT80 & 50\%       & 5     & 344.09                   & 56719.82            & 57063.90  \\ \hline
N2-HT20  & 50\%       & 19    & 2282.28                  & 150459.66           & 152741.95 \\ \hline
N2-HT40  & 50\%       & 19    & 1111.99                  & 153608.15           & 154720.14 \\ \hline
AC-VHT80 & 100\%      & 24    & 688.17                   & 273043.04           & 273731.21 \\ \hline
\end{tabular}%
}
\end{table}

\begin{table}[H]
\caption{Total energy consumption of one month when only both AC-VHT80,N2-HT20 and N2-HT40 compared to only AC-VHT80.}
\resizebox{\textwidth}{!}{%
\begin{tabular}{lll}
WiFi standard     & Total per month (Wh) & \% of energy conserved \\ \hline \hline
AC-VHT80+N2-HT20: & 1772.67              & 23.40                \\ \cline{2-3} 
AC-VHT80+N2-HT40: & 1789.38              & 22.68                \\ \cline{2-3}
AC-VHT80 (100\%): & 2314.28              &                      \\ \cline{2-3} 
\end{tabular}%
}
\end{table}



% \subsection{Efficient WiFi}\label{EfficientWiFi}
% In the previous section \ref{section:SwitchingDynamically} an option that applies older WiFi standard when possible, has been discussed. 
% However, in chapter \ref{researchExperiment} the experiment showed that the newest WiFi standards are most energy efficient. 
% This means that another option to allow the AP to operate more energy efficient is to when possible apply the latest WiFi standard when a connection is made or allow only connections to be made with the latest WiFi standard.

% \subsubsection{Drawbacks latest WiFi standard}
% \begin{enumerate}
%     \item Having a preference in the WiFi standard 
%         \begin{itemize}
%             \item This WiFi network still allows stations to connect with all WiFi standards but prefers the newest standard.
%             Stations that are only capable of the oldest b WiFi standard, will force the other stations to claim the medium using a by b stations decodable control frame. 
%             After claiming the medium, the other stations can transmit their data using their own WiFi standard.
%             When using a hybrid b/g/n WiFi network, this "may cause a significant overhead to network performance" \cite{1460529}.
%         \end{itemize}
% \end{enumerate}

% \begin{enumerate}
%     \item explain what options exist
%     \item the difficulties/problems
%     \item Conclude ?
% \end{enumerate}



% \todo{Change this table}
% \begin{table}[]
% \caption{Total number of Whs consumed by an AP per year when transmitting the weighted amount of data used in 1Q19 (America: 273.5GB, Europe: 175.7GB per month)\cite{OpenVaultReport2019} }
% \centering
% \begin{tabular}{lllll}
% \hline\hline
%       &          & America & Europe  & \% of the oldest standard \\ \cline{2-5} 
% 2.4GHz & B        & 1977.93 & 1270.65 & 100.00                   \\ \cline{2-5} 
%       & G        & 1333.41 & 856.60  & 67.41                    \\ \cline{2-5} 
%       & N2-HT20  & 720.40  & 462.80  & 36.42                    \\ \cline{2-5} 
%       & N2-HT40  & 351.00  & 225.49  & 17.75                    \\ \cline{2-5} 
%       &          &         &         &                          \\ \cline{2-5} 
% 5GHz   & A        & 1001.00 & 643.05  & 100.00                   \\ \cline{2-5} 
%       & N5-HT20  & 540.56  & 347.26  & 54.00                    \\ \cline{2-5} 
%       & N5-HT40  & 273.59  & 175.76  & 27.33                    \\ \cline{2-5} 
%       & AC-VHT20 & 436.04  & 280.12  & 43.56                    \\ \cline{2-5} 
%       & AC-VHT40 & 219.41  & 140.95  & 21.92                    \\ \cline{2-5} 
%       & ACVHT80  & 108.61  & 69.77   & 10.85                    \\ \cline{2-5} 
% \end{tabular}
% \end{table}