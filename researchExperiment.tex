\newcommand{\image}[1]{Images/research/#1}





\section{Measuring energy consumption of WiFi standards}\label{researchExperiment}
\begin{comment}
This chapter, or series of chapters, delves into all technical details that are
required to \emph{prove} your scientific hypothesis.
It should be sufficiently detailed and precise in order for any fellow computing scientist student to be able to \emph{repeat}
your research and therewith establish the same results / conclusions that you have obtained.
Please note that, in order to improve readability of your thesis, you can put a part of this information also in one or
more appendices (see Appendix \ref{appendix}).
\end{comment}


In  this  chapter, the measurement methodology that is used to measure the power consumption of the WiFi standards is described. Next to that results of these measurements will be discussed. 

\subsection{Methodology power consumption measurements}\label{powerConsumptionMeasurement}
\todo{Add that the text file that was used consists of }
The goal is to reduce the energy consumption of a wireless access point. By switching between WiFi standards. To determine of this is a possible approach the energy consumption of the different WiFi standards must be measured. This will be done using an experiment where we will measure the energy consumption of each WiFi standard when retrieving a hosted file. This file will be hosted locally on the network to guarantee identical circumstances among each run of this experiment. \todo{Add that the congestion can not be caused by the ethernet cable} The server that hosts this file is connected to the AP an ethernet cable. The major part of these circumstances that must be identical is file retrieval time, as the energy consumption increases with time. Thus having some retrievals last longer compared to others will have unwanted impact on the results.\\
\\
The file that will be retrieved with a run is one of three different size files, 1MB, 10MB and 100MB. We decided to diversify the sample files to gain data that represents normal use more compared to only one 10MB file. The actual size of the sample files is arbitrary. Other file sizes like 3 MiB would work just as good, but would have been less easy to work with. The sizes could be associated with for example a website which is usually below or around the 1MB in size\cite{websiteSize}, image and video streaming. Next to that, we want to limit the possibility that one standard has unexpected benefits with a specific file size.\\
\\
Measuring the power consumption is done using a tinkerforge voltage/current bricklet v2\footnote{\url{https://www.tinkerforge.com/en/doc/Hardware/Bricklets/Voltage_Current_V2.html}}, which is made up out of an ina 226. The bricklet has a resolution of 1mW, 1mV and 1mA. This is a bricklet for measuring DC energy consumption, meaning that it is placed between the AP and its power plug. The power cable has been cut and spliced open to connect the bricklet. This is beneficial as the power loss due to AC to DC conversion of the power plug is not measured. An overview of the setup can be viewed in figure \ref{fig:experiment-setup}\\

\begin{figure}
    \centering
    \includegraphics[]{\image{experiment-setup.pdf}}
    \caption{Setup of test network}
    \label{fig:experiment-setup}
\end{figure}

The setup in figure \ref{fig:experiment-setup} consist of a few basic components:
\begin{enumerate}
    \item Query Device - This is WiFi capable laptop of up to IEEE 802.11ac
    \item WiFi - This is a WiFi network that is not connected to the world wide web. The wireless network is limited to one WiFi standard at a time. 
    \item Access Point - This is the device that creates the wireless network. In the case of this paper a TP-LINK Archer C7 AC1750 EU v2 running the latest OpenWRT firmware (19.07.2)
    \item Server - this is a computer that is connected to the Access Point using an gigabit cat 6 ethernet cable .
    \item ina 226 voltage current bricklet - this device measures the power consumption of the Access Point. It is controlled by a Master Brick v2.1
\end{enumerate}

\subsection{Results}
In this section, the results from the energy consumption experiments outlined in section \ref{powerConsumptionMeasurement} are presented and examined in detail. The effects of the different WiFi standards on the energy consumption of the AP are explained, and various implications for design are discussed. The raw data can be found in appendix \ref{appendixC}.\newline
\newline
We tested every WiFi standard with 3 different files. The WiFi standards are categorized into two groups. Namely, 2,4GHz(blue) and 5GHz(orange) and arrange from old(left) to new(right). The average energy consumption of the 100mb transfer can be found in figure \ref{fig:results-100mb-energy}\todo{Maybe change graphs time to s ipv ms and J to KJ?}. Next to this, we have the average time \ref{fig:results-100mb-time} the transfer took and the average power consumption\ref{fig:results-100mb-power}.


\begin{figure}
    \centering
    \includegraphics[width=0.95\textwidth]{Images/research/experimentResults/energy-100.pdf}
    \caption{Average energy consumption of each WiFi standard during 100MB test in Joule}
    \label{fig:results-100mb-energy}
\end{figure}

\begin{figure}
    \centering
    \includegraphics[width=0.95\textwidth]{Images/research/experimentResults/time-100.pdf}
    \caption{Average time consumption of each WiFi standard during 100MB test in milliseconds}
    \label{fig:results-100mb-time}
\end{figure}

\begin{figure}
    \centering
    \includegraphics[width=0.95\textwidth]{Images/research/experimentResults/power-100.pdf}
    \caption{Average power consumption of each WiFi standard during 100MB test in milliWatts}
    \label{fig:results-100mb-power}
\end{figure}

\subsubsection{Results?}
We expected that the older WiFi standards would consume less energy. However, looking at figure \ref{fig:results-100mb-energy} an decreasing pattern from older to new is visible in both the 2,4GHz and 5GHz category. We can see that the older WiFi standards actually use more energy when transmitting the same file in combination with the more time it took the older WiFi standards to transmit the text file. However, on the 100MB file it is visible that the average power consumption of the oldest standards (b/a) is lower compared to the other standards. 

\subsubsection{Measurement trouble}
Al thought the measurements went reasonably smooth some problems arose during the experiment. \todo{Maybe this should be moved to section \ref{powerConsumptionMeasurement}}
\begin{itemize}
    \item In one instance the retrieval of the file over the WiFi standard 'b' crashed. The cause of the problem is unknown but was not encountered again in later retrievals.
    \item When sequentially doing the same test the Tinkerforge voltage/current v2 bricklet would not start or end measurement accurately. The retrieval time would doubled with every run. The cause of this is unknown but we suspect it has to do with code optimization. Next to this, it could easily be avoided by altering the program to through out the execution phase of the program only perform one test. This gave more consistent results.
\end{itemize}