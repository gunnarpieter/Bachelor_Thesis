\section{Background IEEE 802.11}\label{preliminaries}
\begin{comment}
This \emph{optional} chapter contains the stuff that your reader needs to know in order to understand
your work.
Your ``audience" consists of fellow third year computing science bachelor students who have done the same
core courses as you have, but not necessarily the same specialization, minor, or free electives.


\begin{itemize}
\item Explain Wi-Fi
\item a frequency-hopping spread-spectrum (FHSS)
\end{itemize}
\end{comment}

In this chapter the important facets of the IEEE 802.11 protocol related to this paper will be explained. The following information is gathered from the IEEE 802.11 specifications, \todo{cite IEEE 802.11 specifications} M. Gast's book "802.11 Wireless Networks The Definitive Guide" and "Computer Networking A Top-Down Approach"
by James F. Kurose and Keith W. Ross \cite{?,GuideWLAN,ATopDownApproach}\\
\\
IEEE 802.11 is a set of medium access control (MAC) and physical layer (PHY) specification for wireless devices. Different versions exist, which can be seen in figure 


\subsection{Network types}
Two different network types exist, an independent network and an infrastructure network. An independent network or ad hoc network is the combination of a set of stations. An example can be see in figure \ref{fig:independent-network}, all mobile stations can communicate with each other. Each packet thus requires one hop. This type of network has a short lifespan, meaning that it is created and then utilized for a small period of time after which the network is dissolved. An infrastructure network is created out of:
\begin{enumerate}
    \item Mobile stations, these are laptops, phones or anything with an wireless network interface controller.
    \item An access point, through which all traffic flows.
\end{enumerate}
Other than in an independent network, an infrastructure network's packets flow through the access point. The communication thus takes two hops. An example can be viewed in figure \ref{fig:infrastructure-network}. A benefit of an infrastructure network is that the mobile stations do not have to be in each others basic service area. A collection of these stations connected to an AP is called a basic service set (BSS). Within this paper we will only be using an infrastructure network as we will be looking more closely at the AP and its energy consumption. 


\begin{figure}
    \begin{minipage}[c]{0.4\linewidth}
        \includegraphics[scale=0.8]{Images/preliminaries/independent-network.png}
        \caption{Independent network example}
        \label{fig:independent-network}
    \end{minipage}
    \hfill
    \begin{minipage}[c]{0.4\linewidth}
        \includegraphics[scale=0.8]{Images/preliminaries/infrastructure-network.pdf}
        \caption{Independent network example}
        \label{fig:infrastructure-network}
    \end{minipage}%
\end{figure}


\subsection{Frame types and their use}
Within IEEE 802.11 multiple types of frames exist, the three most used frames are Data frames, control frames and management frames. These frames have different structure and uses. 

\begin{enumerate}
    \item Data frames - used to transmit data between stations and AP.
    \item Management frames - used to provide authentication, connect and disconnect functionality to wireless networks.
    \item Control frames - these frames help with the transmission of data. All control frames have the same 2 byte Frame Control field which is depicted in figure \ref{fig:bit-frame-control} A few different control frames exsist: 
    \begin{enumerate}
        \item RTS - Request To Send, a broadcasted request from a station signaling that it wants to transmit large data frames to the AP. A threshold is set for which data frame size a RTS should be sent. When granted with a CTS other stations will be silent. Creating a moment where collision is very unlikely.
        \item CTS - Clear To Send, usually a response to a RTS. Giving that station the right of way to transmit large data frames.
        \item ACK - Acknowledgement of received data.
        \item PS-POLL - Power Save Poll, this frame is used by stations that have been in power saving mode. The station transmits this frame to the AP to retrieve possibly buffered frames while the station was in power saving mode. 
    \end{enumerate}
\end{enumerate}

\begin{figure}
    \centering
    \includegraphics[scale=0.8]{Images/preliminaries/bit-Frame-control-diagram.pdf}
    \caption{Frame control fields}
    \label{fig:bit-frame-control}
\end{figure}

\subsection{Network authentication and association}\label{subsection:Network-AUTH-ASS}
\todo{ADD LOOPS TO fig \ref{fig:wifi-auth-ass-states}\\}
\todo{EXPLAIN CLASSES (THE LOOPS)}


\paragraph{Authentication} is the first step that is performed when connecting to a network, either known or unknown.
Two types of authentication exist:
\begin{enumerate}
    \item Open system
    \item Shared key
\end{enumerate}
The open system is used in a WiFi network that is not (password) protected. The second type is a WiFi network that is protected using WEP, WPA or WPA2 usually with a pre-shared keys (PSK). The open system will be discussed as it is sufficient for this paper.
The station starts at state one depicted in figure \ref{fig:wifi-auth-ass-states}. To connect a station to a WiFi network, the station must first authenticate itself to AP it wants to connect to. Because this is an open WiFi network the station does not need a PSK to authenticate. Thus the station sends an authentication request to the AP containing the stations ID which is usually the MAC address, the AP responds with an authentication response with a value of success or failure. When the AP returns success, the station knows that it is now in state two of figure \ref{fig:wifi-auth-ass-states}, the station is authenticated and can proceed to association.

\paragraph{Association} is usually performed directly after authentication, the AP and station know each others identity and the station sends a simple association request. 
The way an AP decides when to allow a station to associate with the network is not written in stone within IEEE 802.11. When an AP decides the station will be allowed to associate, it sends an association response with the status code of success (0) and the Association ID (AID) for the station. This AID is used to identify the station when frames have been buffered while the stations was in power saving mode. 
When the station received this response it knows it is now in state 3. Meaning it can start to transfer data frames with the AP.

\begin{figure}
    \centering
    \includegraphics[scale=0.8]{Images/preliminaries/wifi-auth-ass-states.pdf}
    \caption{IEEE 802.11 authentication and association states}
    \label{fig:wifi-auth-ass-states}
\end{figure}

\subsection{Power conservation}\label{section:PowerManagement}
Power management is currently implemented in such a way that it heavily utilizes the access point. When a station decides it wants to sleep or enter Power Saving mode (PS) it indicates this to the AP. 
The AP performs two tasks to provide the stations with PS capability, the first is buffering the frames that arrive for a station that is currently sleeping. 
Secondly, within the beacon the AP broadcasts periodically, the AP includes the stations AID for which it has buffered frames. 
An important limitation is that the AP has an finite amount of memory and thus the AP and station agree to a 'Listen Interval'. The AP waits at least the Listen Interval before it discards any frames.
The current "Power management [system] is designed around the needs of the battery-powered mobile stations" \cite{GuideWLAN}.\\
Transmit Power Control (TPC) limits the transmission power such that it stays within the regulatory required ranges\cite{GuideWLAN}. 
Transmitting a signal costs energy, if a signal can reaches further than what it needs to, the energy used to transmit it further is wasted. 
TPC was not intended to be used to conserve power but the papers \cite{6983074} and \cite{1400406} both use this feature to reduce the energy consumption.


\subsection{IEEE 802.11 b/a/g/n/ac/ax}
The first WiFi version became available to consumers in 1997 it was called IEEE 802.11. Over time the standard received updates, usually becoming faster as the original IEEE 802.11 supported only speeds of up to 2Mbp/s. These updates are called amendments and the first two amendments were both released in 1999, IEEE 802.11b IEEE 802.11a. IEEE 802.11b uses a 2.4GHz frequency and has a theoretical limit of 11Mbp/s, IEEE 802.11b does not bring new features to the WiFi family, it is merely an update of the first standard. IEEE 802.11a on the other hand brought orthogonal frequency division multiplexing (OFDM) to the table. Next to this, IEEE 802.11a operates on the 5GHz frequency band allowing a max theoretical max throughput of 54 Mbp/s.\newline
In 2003 faster 2.4GHz standard was released, IEEE 802.11g. IEEE wanted "to combine the best of both 802.11a and 802.11b" \footnote{\url{https://www.lifewire.com/wireless-standards-802-11a-802-11b-g-n-and-802-11ac-816553}}. That is why, 802.11g is backwards compatible with IEEE 802.11b and also implements OFDM achieving a theoretical max throughput of 54Mbp/s. The 2.4GHz devices are cheaper to manufacture\footnote{https://alethea.in/iot-age-wifi-6/} compared to 5GHz.
Only 6 years later IEEE 802.11n was released bringing Multiple Input Multiple Output (MIMO) with it. Simply put MIMO, uses several antennas to send and receive more data at a time, achieving a theoretical max speed of 600Mbp/s using MIMO and 300Mbp/s without\footnote{\url{https://www.intel.com/content/www/us/en/support/articles/000005714/network-and-i-o/wireless-networking.html}}. The IEEE 802.11n works on both 2.4GHz and 5GHz frequencies.
