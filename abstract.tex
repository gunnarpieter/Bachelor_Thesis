\begin{abstract}
% The abstract of your thesis is a brief description of the research hypothesis,
% scientific context, motivation, and results.
% The preferred size of an abstract is one paragraph or one page of text.
\noindent
The Dutch government has made new goals with the ICT sector to decrease the burden on our planet. 
The aim for the ICT sector is to be 50\% more energy efficient in 2030 then in 2005. 
This thesis investigates the possibilities of increasing the energy efficiency of an access point by using the WiFi standards b/a/g/n/ac. 
An experiments were conducted to establish the power consumption of an Archer C7 AC1750 access point. Next to this, simulations were performed to get insight in the energy consumption of this access point within two three scenarios: normal use, wake/sleep cycle and switching between WiFi standards.
%results:
The experiment performed showed that the older WiFi standards have a lower power draw but due to a longer transmission time end up consuming more electricity while transferring a file. Next to this, the idle power consumption of older WiFi standards are less compared to newer WiFi standards. However, the idle power draw results can be better categorized by 2.4GHz and 5GHz with exception of 802.11 a and n with a 20MHz bandwidth. The first simulation shows that 802.11n with a 20MHz bandwidth consumes the least amount of energy. However, 802.11ac with a 80MHz bandwidth becomes the least consuming standard when the network is off if it is not in use.\\
The second simulation of switching between 802.11 2.4GHz (g,n2-ht20,n2-HT40) and 802.11ac with a 80MHz bandwidth WiFi network shows roughly an 23\% decrease in energy consumption of the access point when switching between two standards. We conclude that a dynamic/static switching method can significantly reduce the energy consumption of an access point.

\end{abstract}

