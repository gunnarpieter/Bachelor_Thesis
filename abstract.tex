\begin{abstract}
% The abstract of your thesis is a brief description of the research hypothesis,
% scientific context, motivation, and results.
% The preferred size of an abstract is one paragraph or one page of text.
\noindent
%Problem
The Dutch government has made new goals with the ICT sector to decrease the burden on our planet. 
The aim for the ICT sector is to be 50\% more energy efficient in 2030 then in 2005. 
%onderzoek
This thesis wants to contribute by investigating the possibilities of increasing the energy efficiency of an access point using the WiFi standards 802.11 b/a/g/n/ac with their respective bandwidths. The bandwidth is a specific range of frequencies that the AP uses to transfer data on and in this case the bandwidth can either be 20, 40 or 80MHz. A larger bandwidth allows for more data to be transferred but also has a higher risk of interference with other WiFi devices.\\
%opzet
An experiment was conducted to establish the power consumption of an Archer C7 AC1750 access point. Next to this, three simulations were performed using the information gather from the experiment to get insight in the energy consumption of this access point. 
%The first simulation is of the normal operation of the access point. The second simulation is of a strict wake/sleep cycle. The AP is only on when transmitting data. The last simulation utilises switching between 802.11 2.4GHz (g, n2-ht20, n2-HT40) and 802.11ac with a 80MHz bandwidth. For this simulation some assumption had to be made which are listed in section \ref{outcomeSwitching} \\
%results:
\\The experiment performed shows that the b, a and g standards draw less power compared to the their 2.4GHz and 5GHz counter parts but due to a longer transmission time the total consumed amount of energy is significantly larger. Moreover, all 2.4GHz networks have a similar idle power draw. Within the 5GHz category the a and n with a 20MHz standard also have similar idle power draw. The remaining 5GHz standards draw around the $3$W.\\
The first two simulations showed that 802.11n with a 20MHz bandwidth consumes the least amount of energy with a monthly consumption. However, 802.11ac with a 80MHz bandwidth becomes the least consuming standard when the network is turned off if it is not in use.\\
The final simulation shows roughly an 23\% decrease in energy consumption of the access point when switching between two standards. With similar results when switching between g compared to n with a 20MHz bandwidth and n with a 40MHz bandwidth.\\
%conclusion
\\
Based on the results of the simulation, the most energy efficient standard is 2.4GHz 802.11n at a 20MHz bandwidth. Using this standard will result in the largest energy reduction compared to using 802.11ac with a 80MHz bandwidth. Next to this, a wake/sleep solution can be applied to reduce the energy consumption. The same holds for a dynamic/static switching method. Which has been shown to significantly reduce the energy consumption of an access point.

\end{abstract}

% The Dutch government has made new goals with the ICT sector to decrease the burden on our planet. 
% The aim for the ICT sector is to be 50\% more energy efficient in 2030 then in 2005. 
% %onderzoek
% This thesis wants to contribute by investigating the possibilities of increasing the energy efficiency of an access point using the WiFi standards 802.11 b/a/g/n/ac with their respective bandwidths. The bandwidth is a specific range of frequencies that the AP uses to transfer data on and in this case the bandwidth can either be 20, 40 or 80MHz. A larger bandwidth allows for more data to be transferred but also has a higher risk of interference with other WiFi devices.\\
% %opzet
% An experiment was conducted to establish the power consumption of an Archer C7 AC1750 access point. Next to this, three simulations were performed using the information gather from the experiment to get insight in the energy consumption of this access point. 
% %The first simulation is of the normal operation of the access point. The second simulation is of a strict wake/sleep cycle. The AP is only on when transmitting data. The last simulation utilises switching between 802.11 2.4GHz (g, n2-ht20, n2-HT40) and 802.11ac with a 80MHz bandwidth. For this simulation some assumption had to be made which are listed in section \ref{outcomeSwitching} \\
% %results:
% \\The experiment performed shows that the b and g WiFi standards draw between $0.17$ and $0.83$ less power but due to a longer transmission time b and g consume $3.69$ and $2.49$ times as much energy as the average of the 2.4GHz 802.11n standard. On the 5GHz frequency 802.11a consumes $3.17$ times as much energy as the average of all other 5GHz standards but draws between $0.20$ and $0.83$W less power when idling. Next to this, the idle power consumption of the 2.4GHz networks deviates at most 0.04W.

% Moreover, the idle power draw results can be better categorized by 2.4GHz and 5GHz with exception of 802.11 a and n with a 20MHz bandwidth. The largest difference in the 2.4GHz standards is $0.04$W when idling, this is $0.01$W for the 5GHz standards 802.11a and 802.11n with a 20MHz bandwidth. The remaining 5GHz standards are all around the $3$W.\\
% The first two simulations showed that 802.11n with a 20MHz bandwidth consumes the least amount of energy with a monthly consumption of $1636.53$Wh. However, 802.11ac with a 80MHz bandwidth becomes the least consuming standard when the network is turned off if it is not in use. This standard then consumes $1574.97$Wh per month.\\
% The final simulation shows roughly an 23\% decrease in energy consumption of the access point when switching between two standards. With similar results when switching between g compared to n with a 20MHz bandwidth and n with a 40MHz bandwidth.\\
% %conclusion
% \\
% Based on these results, the most energy efficient standard is 2.4GHz 802.11n at a 20MHz bandwidth. Only using this standard will result in the largest energy reduction compared to using 802.11ac with a 80MHz bandwidth. Next to this, a wake/sleep solution can be applied to reduce the energy consumption. The same holds for a dynamic/static switching method. Which has been shown to significantly reduce the energy consumption of an access point.