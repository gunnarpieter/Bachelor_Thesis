\section{Conclusions}\label{conclusions}
In this chapter you present all conclusions that can be drawn from the
preceding chapters.
It should not introduce new experiments, theories, investigations, etc.:
these should have been written down earlier in the thesis.
Therefore, conclusions can be brief and to the point.



This thesis conducted an experiment into the energy consumption of the IEEE 802.11 WiFi standards b/a/g/n/ac. 
Next to this, it looked more closely at switching dynamically between these standards as a way of saving energy and applying the knowledge gained about the standards in an effective manner. 
\newline
\newline
Voor eerste sub vraag (a):
We can see that the older WiFi standards actually use more energy when transmitting the same file in combination with the more time it took the older WiFi standards to transmit the text file.  Interestingly, on the 100MB file test it is visible that the aver-age power consumption of the oldest standards (b/a) is lower compared to the other standards.
\newline
\newline
(b i): 
The results from the experiment performed in section 3 have shown that dynamically switching, will not lead to a reduction in the energy consumption of the access point but to an increase due to the extra time it takes for the data to be transferred which vanquishes the small amount of power draw difference.
\newline
\newline
(b ii):
Moet nog geschreven worden. Komt uiteindelijk een conclusie hoeveel procent de ene zuiniger is dan de andere. en hoeveel tijd er in idle/slaap gespendeerd kan worden.
\newline
\newline
(b): Dit moet ook nog geschreven worden maar er moet in \ref{EfficientWiFi} bij b ii ook iets komen dat de nieuwste standaard sneller is en dat dit er dus voor zorgt dat dynamische algoritmen (1 paper besproken in related work) meer profijt kan hebben door dan langer te slapen.

\begin{enumerate}
    \item Welke mogelijkheden zijn er om een Access Point energiezuinger te laten functioner met gebruik van de WiFi standaarden?
     \begin{enumerate}
        \item Wat is het energieverbruik van elke WiFi standaard?
        \item Welke toepassingsmogelijkheden van WiFi standaarden sluiten aan bij een besparing van energie?
            \begin{enumerate}
                \item Kan dynamisch switchen tussen WiFi standaarden energie besparen?
                \item Welke WiFi standard is het meest efficient qua energie verbruik?
            \end{enumerate}
    \end{enumerate}
\end{enumerate}

\subsection{Future work}