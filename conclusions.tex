\chapter{Conclusions}\label{conclusions}
% In this chapter you present all conclusions that can be drawn from the
% preceding chapters.
% It should not introduce new experiments, theories, investigations, etc.:
% these should have been written down earlier in the thesis.
% Therefore, conclusions can be brief and to the point.



This thesis conducted an experiment into the energy consumption of the IEEE 802.11 WiFi standards b/a/g/n/ac on an Archer AC1750 access point. 
Several measurements have been performed to achieve this, namely a file transfer using a 1, 10 and 100mb file and an idle measurement. Every measurement has been performed in a low-radiation environment. Meaning no presence of cellular signal or other WiFi networks.
Next to this, it looked more closely at switching between these standards as a way of saving energy and applying the knowledge gained about the standards in an attempt to curb the growth rate of the electricity used by networks. 
\newline
\newline
%In section \ref{researchExperiment} the information to answer the question 'What is the energy consumption of each WiFi standard?' was gathered. 
The experiment performed to gather the information regarding the energy consumption of the access point resulted in several interesting findings. Specifically that, while transmitting the power draw of the older 2.4GHz b and g standard is $0.17$ to $0.36$W lower compared to both 20MHz and 40MHz bandwidths on 2.4GHz 802.11n and similar holds for the 5GHz standards. Where 802.11a is $0.83$ and $0.46$ lower compared to 802.11n with a 40MHz bandwidth and 802.11ac with a 80MHz bandwidth respectively. 
However, due to slower transfer rates of the older standards standards they consume more energy when transmitting the same file. Next to this the idle power draw experiment showed that the idle power draw of the 2.4GHz networks are within $0.04$W of each other. Within the 5GHz networks a larger difference can be found. 802.11n with a 40MHz bandwidth and all 802.11ac networks draw at least $0.44$W more compared to the networks running 802.11a and 802.11n with a 20MHz bandwidth.\newline
\newline
There is a specific WiFi standard that consumes the least amount of energy when looking only at the power draw of transmitting or idle one might expect that 802.11b, 802.11g or 802.11a would be the most efficient. However, table \ref{table:EnergyConAP} shows clearly that 2.4GHz 802.11n with a 20MHz bandwidth consumes $1636.53$Wh which is the least amount of energy of all standards, if the network remains active. This result changes when the total energy consumption is calculated when the AP turns of if not in use. In this case the newest 802.11ac with a 80MHz bandwidth standard becomes the most energy efficient standard with a total of $1574.97$Wh instead of $2312.78$. This difference comes from a high idle power draw combined with the fast transmissions rates of 802.11ac lengthening the time in which the AP can turned off. Based on these conclusions, the best reduction can be achieved when only using 2.4GHz 802.11n at a 20MHz bandwidth.
\newline
\newline
%(b ii): 
The results described in section \ref{section:SwitchingDynamically} show that dynamically/statically switching can lead to a reduction in the energy consumption of roughly 23\% of an access point. 
However, a key point is that the reduction will not come from less energy consumption during transmission of data but from the difference in idling power draw of each WiFi standard. 
Next to this, it is apparent that the switch should also not be between old and new but 2.4GHz and 5GHz. 
This can be concluded from table \ref{table:MonthEnergyCons} where the percentage of energy conserved by the switch to 802.11g is lower but close to the same switch but then to 802.11n with a 20MHz bandwidth. 
Next to this, the transmission time of the data almost doubles between these different standards. Impacting user comfort and the usability of the WiFi network.\newline
\newline
In this thesis it has been shown that a dynamic/static switching method can help to reduce the energy consumption of an access point. Next to this, the results in this paper support the solution of utilizing a wake/sleep cycle which has been applied in papers \cite{6983074,4489442,pscBatteryWLAN,7502830}. 

% \begin{enumerate}
%     \item Welke mogelijkheden zijn er om een Access Point energiezuinger te laten functioner met gebruik van de WiFi standaarden?
%      \begin{enumerate}
%         \item Wat is het energieverbruik van elke WiFi standaard?
%         \item Welke toepassingsmogelijkheden van WiFi standaarden sluiten aan bij een besparing van energie?
%             \begin{enumerate}
%                 \item Welke WiFi standard is het meest efficient qua energie verbruik?
%                 \item Kan dynamisch switchen tussen WiFi standaarden energie besparen?
%             \end{enumerate}
%     \end{enumerate}
% \end{enumerate}

\section{Future work}
Archer AC1750 -> Meer effect op higher end models?
Test setup dus onderzoek naar verbuik met echt realistisch gebruik.
Meer informatie over gebruik van de WiFi standaard.